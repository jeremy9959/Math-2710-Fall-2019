\PassOptionsToPackage{unicode=true}{hyperref} % options for packages loaded elsewhere
\PassOptionsToPackage{hyphens}{url}
%
\documentclass[ignorenonframetext,]{beamer}
\usepackage{pgfpages}
\setbeamertemplate{caption}[numbered]
\setbeamertemplate{caption label separator}{: }
\setbeamercolor{caption name}{fg=normal text.fg}
\beamertemplatenavigationsymbolsempty
% Prevent slide breaks in the middle of a paragraph:
\widowpenalties 1 10000
\raggedbottom
\setbeamertemplate{part page}{
\centering
\begin{beamercolorbox}[sep=16pt,center]{part title}
  \usebeamerfont{part title}\insertpart\par
\end{beamercolorbox}
}
\setbeamertemplate{section page}{
\centering
\begin{beamercolorbox}[sep=12pt,center]{part title}
  \usebeamerfont{section title}\insertsection\par
\end{beamercolorbox}
}
\setbeamertemplate{subsection page}{
\centering
\begin{beamercolorbox}[sep=8pt,center]{part title}
  \usebeamerfont{subsection title}\insertsubsection\par
\end{beamercolorbox}
}
\AtBeginPart{
  \frame{\partpage}
}
\AtBeginSection{
  \ifbibliography
  \else
    \frame{\sectionpage}
  \fi
}
\AtBeginSubsection{
  \frame{\subsectionpage}
}
\usepackage{lmodern}
\usepackage{amssymb,amsmath}
\usepackage{ifxetex,ifluatex}
\usepackage{fixltx2e} % provides \textsubscript
\ifnum 0\ifxetex 1\fi\ifluatex 1\fi=0 % if pdftex
  \usepackage[T1]{fontenc}
  \usepackage[utf8]{inputenc}
  \usepackage{textcomp} % provides euro and other symbols
\else % if luatex or xelatex
  \usepackage{unicode-math}
  \defaultfontfeatures{Ligatures=TeX,Scale=MatchLowercase}
\fi
% use upquote if available, for straight quotes in verbatim environments
\IfFileExists{upquote.sty}{\usepackage{upquote}}{}
% use microtype if available
\IfFileExists{microtype.sty}{%
\usepackage[]{microtype}
\UseMicrotypeSet[protrusion]{basicmath} % disable protrusion for tt fonts
}{}
\IfFileExists{parskip.sty}{%
\usepackage{parskip}
}{% else
\setlength{\parindent}{0pt}
\setlength{\parskip}{6pt plus 2pt minus 1pt}
}
\usepackage{hyperref}
\hypersetup{
            pdftitle={Math 2710},
            pdfauthor={Aug 26-28},
            pdfborder={0 0 0},
            breaklinks=true}
\urlstyle{same}  % don't use monospace font for urls
\newif\ifbibliography
\setlength{\emergencystretch}{3em}  % prevent overfull lines
\providecommand{\tightlist}{%
  \setlength{\itemsep}{0pt}\setlength{\parskip}{0pt}}
\setcounter{secnumdepth}{0}

% set default figure placement to htbp
\makeatletter
\def\fps@figure{htbp}
\makeatother


\title{Math 2710}
\author{Aug 26-28}
\date{}

\begin{document}
\frame{\titlepage}

\hypertarget{course-info}{%
\section{Course Info}\label{course-info}}

\begin{frame}{Key links}
\protect\hypertarget{key-links}{}

\begin{itemize}
\tightlist
\item
  \href{https://learn.uconn.edu/bbcswebdav/courses/M1198-MATH-2710-001.002/syllabus.pdf}{Syllabus}
\item
  Tests
\item
  \href{https://learn.uconn.edu/bbcswebdav/courses/M1198-MATH-2710-001.002/Homework.html}{Homework}
\item
  \href{https://piazza.com/uconn/fall2019/m1198math2710001002}{Piazza}
\end{itemize}

\end{frame}

\begin{frame}{Grading}
\protect\hypertarget{grading}{}

\begin{itemize}
\tightlist
\item
  Two midterms (25 points) tentatively Sep 30 and Nov 5.

  \begin{itemize}
  \tightlist
  \item
    Notify me by Sep 20 if you need an alternate date for the first exam
    because of Rosh Hashanah.
  \end{itemize}
\item
  Final Exam (40 points)
\item
  Homework (8 points)
\item
  Piazza participation (2 points)
\end{itemize}

\end{frame}

\begin{frame}{Homework}
\protect\hypertarget{homework}{}

\begin{itemize}
\tightlist
\item
  daily assignments
\item
  periodically collected and graded with short lead time
\item
  assorted short quizzes or other assignments from time to time
\end{itemize}

\end{frame}

\hypertarget{what-is-this-course-about}{%
\section{What is this course about?}\label{what-is-this-course-about}}

\begin{frame}{Mathematics as a discipline}
\protect\hypertarget{mathematics-as-a-discipline}{}

This course is about

\begin{itemize}
\tightlist
\item
  \emph{how mathematics is done}
\item
  \emph{how mathematics is communicated}.
\end{itemize}

The actual mathematics we will learn in this course is less important
than the approach

\end{frame}

\begin{frame}{A very simple example}
\protect\hypertarget{a-very-simple-example}{}

\textbf{Assertion:} The sum of two even numbers is an even number.

Question: why is this true?

\end{frame}

\begin{frame}{Mathematical Proof}
\protect\hypertarget{mathematical-proof}{}

A \emph{mathematical proof} of this assertion is an argument that starts
from known facts and definitions and establishes the the truth of the
assertion using the tools of logic.

A good mathematical proof is

\begin{itemize}
\tightlist
\item
  \emph{rigorous}, meaning it gives a complete logical argument,
\item
  \emph{informative}, meaning that it provides enough information to
  explain why the assertion is true
\item
  \emph{efficient}, meaning that it is as short as possible while still
  being rigorous and informative.
\end{itemize}

\end{frame}

\begin{frame}{Example, continued}
\protect\hypertarget{example-continued}{}

To construct a proof of this assertion, we need:

\begin{itemize}
\tightlist
\item
  to know exactly what the terms mean (what is an even integer?)
\item
  to establish in our own minds that the assertion IS true, and figure
  out why
\item
  communicate our understanding of why the assertion is true rigorously
  and efficiently.
\end{itemize}

\end{frame}

\begin{frame}{Discussion}
\protect\hypertarget{discussion}{}

\begin{itemize}
\tightlist
\item
  Define \emph{even number}.
\item
  Explain why the assertion about even numbers is true, as rigorously
  and efficiently as you can.
\end{itemize}

\end{frame}

\begin{frame}{Key Vocabulary}
\protect\hypertarget{key-vocabulary}{}

theorem, lemma, proposition, corollary, example, algorithm, definition,
proof, statement, proposition, converse, contrapositive, conditional
statement

\end{frame}

\end{document}

%%% Local Variables:
%%% mode: latex
%%% TeX-master: t
%%% End:
